\section{Koncepcja}
\subsection{Zleceniodawca}
Nasz projekt będzie realizowany na zlecenie firmy North Press, która zajmuje się
prowadzeniem sieci salonów prasowych Świat Prasy. Potrzebuje ona systemu umożliwiającego
sprawne monitorowanie sprzedaży i~stanu magazynów, ułatwiającego podejmowanie decyzji w~kwestiach
logistyki, a~także wspomagającego generowanie zamówień do dostawców, faktur dla
klientów oraz dziennych raportów ze sprzedaży.

\subsection{Ogólna wizja projektu}
System, który zaprojektujemy, ma za zadanie zbierać dane dotyczące sprzedaży ze wszystkich
stanowisk kasowych (używane w~tej chwili kasy fiskalne umożliwiają przesyłanie danych pomiędzy
kasą, a podłączonym do niej komputerem) i~umożliwiać ich analizę kierownictwu firmy, a~także
samym kasjerom, jak również ułatwiać im podejmowanie decyzji dotyczących zamawiania i~składowania
towaru\footnote{Dość niefortunne określenie dla książek i~czasopism, ale bierzemy pod uwagę,
że w~salonach sprzedawane są także papierosy, gumy do żucia, kupony pre-paid, bilety MPK, etc.}
na podstawie uzyskanych danych.

Podstawowe funkcje, które system powinien spełniać, to:
\begin{itemize}
    \item zbieranie i~przechowywanie danych dotyczących sprzedaży mających miejsce we wszystkich salonach
    \item zbieranie i~przechowywanie danych dotyczących towaru składowanego we wszystkich salonach,
    z~możliwością wglądu również dla kasjerów (w celu ułatwienia im sprawdzenia, czy dany towar jest
    możliwy do nabycia w~tym lub innym salonie, a~także w~jakiej ilości jest tam przechowywany)
    \item generowanie raportów sprzedaży
    \item generowanie faktur dla klientów oraz księgowanie faktur wystawionych przez dostawców.
    \item analiza wydajności poszczególnych pracowników i~salonów (również z~uwzględnieniem określonych przedziałów czasowych)
    \item analiza popytu na poszczególne towary w~salonach
    \item sugerowanie (po analizie popytu i~sprzedaży) wysokości zamówień do dostawców na poszczególne towary oraz generowanie tychże
    \item sugerowanie (po analizie popytu) miejsca i~proporcji przechowywania poszczególnych
    towarów (np.\ 200 sztuk ,,Przyjaciółki'' w~salonie nr 1, 150 sztuk w salonie nr 2, 40 w salonie
    nr 3), z~uwzględnieniem możliwości przechowywania w~każdym z~miejsc branych pod uwagę
    \item sugerowanie (po analizie sprzedaży i~stanu poszczególnych magazynowego w~salonach w~chwili bieżącej)
    przetransportowania określonej ilości towaru do danego salonu, aby zdołał on zaspokoić popyt
    \item sugerowanie (po analizie wydajności) wprowadzenia zmian w~godzinach otwarcia poszczególnych salonów
\end{itemize}

Rozważamy też rozszerzenie projektu o~dodatkowe funkcje, jak na przykład system punktowy dla
stałych klientów lub możliwość prenumeraty czasopism i~wydawnictw cyklicznych (z~możliwością
wysyłki pocztą/kurierem lub odbioru osobistego). Decyzja na ten temat zostanie podjęta po
rozmowach z~zleceniodawcą.

