\section{Dziedzina problemu}
Opracowywany system informatyczny przeznaczony jest dla sieci salonów prasowych Świat Prasy będących własnością firmy North Press. W~chwili obecnej (listopad 2007) sieć obejmuje województwa Kujawsko-Pomorskie, Pomorskie oraz Warmińsko-Mazurskie. System będzie używany przez pracowników zarządzających przedsiębiorstwem oraz obsługę salonów (w~szczególności kierowników salonu). Towar dostarczany jest bezpośrednio do salonu, a~jego obsługa zobowiązana jest do dokonania w~systemie odpowiednich czynności związanych z~tą dostawą.

Pracowników przedsiębiorstwa można podzielić na trzy główne grupy: zarządzający, kierownicy saloników, pracownicy saloników.

Zadaniem zarządzających jest obsługa (wprowadzanie, usuwanie, zmiana) informacji dotyczących towarów sprzedawanych w~salonikach, w~szczególności ustalanie ceny, stawek podatku, kodów kreskowych PLU (ang.\ \emph{Price Look-Up}) oraz przynależność towarów do odpowiedniej grupy towarów; zarządzanie ofertami konkretnych saloników, w~tym ustalanie rzeczywistej ceny sprzedaży w danym salonie, upustów, rabatów oraz określenie dostawcy danego towaru do danego saloniku. Zarządzający mają wgląd do dokumentów generowanych przez salony, lecz nie modyfikują ich. Ponadto zarządzający otrzymują informacje na temat sprzedaży i~obrotów w~każdym z~saloników.

Pracownicy salonów mają możliwość zmienić ustaloną przez zarząd cenę danego towaru (np.\ gazeta przychodzi od dostawcy z~nadrukowaną inną ceną niż miała dotychczas), lecz o~tej zmianie zobowiązani są poinformować zarządzającego. Zarządzający potwierdzą tę zmianę lub odrzuca. W przypadku potwierdzenia może również podjąć decyzje o~zmianie ceny w~pozostałych salonach. Pracownicy salonów mogą zmieniać też kody PLU zapisane w~kasie (np.\ z~powodu zmiany kodu artykułu) o~czym również informują zarządzającego. Pracownik przyjmujący dostawę towaru zobowiązany jest udokumentować ją tworząc dokument przyjęcia do magazynu. Dokument ten dostępny jest do wglądu dla kierownika salonu oraz dla zarządu. Generowane są także dokumenty wydania z~magazynu (np.\ w~przypadku zwrotów prasy do dostawcy). Jeżeli dostawca razem z~dostawą pozostawia w~salonie fakturę, faktura ta musi być przez pracownika salonu przekazana osobie zarządzającej. Po zakończeniu zmiany w~salonie dokonywane jest rozliczenie tej zmiany zawierające informacje na temat stanu kasy na zakończenie (ilość banknotów każdego rodzaju, wartość bilonu, ilość i~wartość płatności kartą, wartość zasiłku zostawionego na kolejną zmianę). Ponadto kierownik salonu dysponuje analogicznymi informacjami na temat sprzedaży i~obrotów jak zarząd, jednakże dotyczącymi wyłącznie jego salonu.

Ponadto od czasu do czasu pracownicy zarządzający wykonują również zadania przeznaczone z założenia dla pracowników salonu.

Każdy z~salonów działa niezależnie od pozostałych, tzn.\ zwykle nie przenosi się towaru, ani środków finansowych pomiędzy nimi, aczkolwiek czasami dokonuje się operacji przeniesienia towaru. Wówczas tworzone są dokumenty wydania i~przyjęcia do magazynu.

Każdy z~salonów dysponuje połączeniem z~internetem (ADSL lub GPRS) oraz kasą fiskalną \textbf{JAKA KASA???} wraz z~czytnikiem kodów kreskowych.
%TODO wybrać kasę

