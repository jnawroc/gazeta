\section{Słownik pojęć roboczych}
\begin{description}
\item[Kierownik] - pracownik odpowiedzialny za obsługę (wprowadzanie, usuwanie, zmianę) informacji dotyczącej towarów sprzedawanych w salonikach. W szególności ustalanie ceny, stawek podatku, kodów kreskowych. Kierownik ma wgląd do dokumentów generowanych przez salony. Ponadto otrzymujr informacje na temat sprzedaży i obrotów w każdym z saloników.
\item[Kierownik salonu] - pracownik zarządzający konkretnym salonem. Ma wgląd do dokumentów generowanych przez zarządzany przez niego salon.
\item[PLU] - (ang. \emph{Price Look-Up}) kod kreskowy, szeroko używany w sklepach i logistyce do szybkiej i automatycznej identyfikacji produktów.
\item[Pracownik] - pracuje w salonie, rejestruje sprzedaże, przyjmuje dostawy i wydaje zwroty.
\item[Towar] - produkty sprzedawane przez saloniki, między innymi: książki, czasopisma, papierosy, gumy do żucia, kupony pre-paid, bilety autobusowe.
\end{description}

